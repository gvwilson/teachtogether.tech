\chapter{Código de Conducta}\label{s:conduct}

Con el objetivo de fomentar un ambiente abierto y amigable, las personas 
a cargo, nos comprometemos a hacer de la participación en nuestro proyecto 
y nuestra comunidad una experiencia libre de acoso para todas las personas, 
independientemente de su edad, tama\~no corporal, discapacidad, etnia, 
identidad y expresión de género, nivel de experiencia, educación, 
nivel socioeconómico, nacionalidad, apariencia personal, raza, 
religión o identidad y orientación sexual.

\section*{Nuestros estándares}

Ejemplos de comportamiento que contribuyen a crear un ambiente positivo 
para nuestra comunidad:

\begin{itemize}
\item
  utilizar un lenguaje amigable e inclusivo, 
\item
  respetar diferentes opiniones, puntos de vista y experiencias,
\item
  aceptar adecuadamente la crítica constructiva,
\item
  enfocándose en lo que es mejor para la comunidad y
\item
  mostrando empatía hacia otros miembros de la comunidad.
\end{itemize}

\noindent
Ejemplos de comportamiento inaceptable:

\begin{itemize}
\item
  el uso de lenguaje o imágenes sexualizadas como también 
  atención o avances sexuales no deseados,
\item
  comentarios despectivos (trolling), insultantes y ataques personales o políticos,
\item
  cualquier tipo de acoso en público o privado,
\item
  Publicar información privada de otras personas, tales como direcciones 
  físicas o de correo electrónico, sin su permiso explícito
\item
  Otras conductas que puedan ser razonablemente consideradas 
  como inapropiadas en un entorno profesional
\end{itemize}

\section*{Nuestras responsabilidades}

Las personas encargadas el proyecto somos responsables de aclarar los estándares de
comportamiento aceptable y se espera que tomemos medidas de acción correctiva 
apropiadas y justas en respuesta a cualquier caso de comportamiento que 
consideremos inaceptable.

Las personas encargadas del proyecto tienen el derecho y la responsabilidad de 
eliminar, editar o rechazar comentarios, commits, código, ediciones wiki, issues y otras
contribuciones que no están alineadas con este código de conducta. También pueden 
prohibir la participación temporal o permanente de cualquier persona por comportamientos 
que consideremos inapropiados, amenazantes, ofensivos o da\~ninos.

\section*{Alcance}

Este código de conducta se aplica dentro de los espacios del proyecto 
y en espacios públicos, cuando una persona representa al proyecto o a
la comunidad. Ejemplos de representación del proyecto o la comunidad incluyen
el uso de una dirección de correo electrónico oficial del proyecto, 
realizar publicaciones a través de una cuenta oficial de redes sociales, 
o actuar como representante oficial en cualquier tipo de evento. 
La representación del proyecto puede ser aclarada y definida en más
detalles por las personas encargadas.

\section*{Aplicación}

Los casos de comportamiento abusivo, acosador o inaceptable 
pueden ser informados enviando un correo electrónico a \texttt{gvwilson@third-bit.com}. 
Todas las quejas serán revisadas e investigadas y darán como resultado 
la respuesta que se considere necesaria y apropiada a las circunstancias.
El equipo encargado del proyecto está obligado a mantener la privacidad de 
quienes reporten incidentes.  Se pueden publicar por separado más detalles 
de políticas de aplicación específicas.

Las personas encargadas del proyecto que no cumplan o hagan cumplir 
de buena fe este código de conducta pueden enfrentar repercusiones 
temporales o permanentes determinadas por el resto del equipo encargado 
del proyecto.

\section*{Atribución}

Este código de conducta es una adaptación del 
\hreffoot{https://www.contributor-covenant.org}{Contributor Covenant} version 1.4.
