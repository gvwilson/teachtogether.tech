\chapter{Teamwork Rubric}\label{s:peereval}

This rubric is designed to assess individual performance within a
team. You can use it as a starting point for creating a rubric of
your own. Rate each item as ``Yes'', ``Iffy'', ``No'', or ``Not Applicable''.

\begin{itemize}
\tightlist
\item
  Communication

  \begin{itemize}
  \tightlist
  \item
    Listens attentively to others without interrupting.
  \item
    Clarifies with others have said to ensure understanding.
  \item
    Articulates ideas clearly and concisely.
  \item
    Gives good reasons for ideas.
  \item
    Wins support from others.
  \end{itemize}
\item
  Decision Making

  \begin{itemize}
  \tightlist
  \item
    Analyzes problems from different points of view.
  \item
    Applies logic in solving problems.
  \item
    Offers solutions based on facts rather than ``gut feel'' or intuition.
  \item
    Solicits new ideas from others.
  \item
    Generates new ideas.
  \item
    Accepts change.
  \end{itemize}
\item
  Collaboration

  \begin{itemize}
  \tightlist
  \item
    Acknowledges issues that the team needs to confront and resolve.
  \item
    Encourages ideas and opinions even when they differ from his/her own.
  \item
    Works toward solutions and compromises that are acceptable to all involved.
  \item
    Shares credit for success with others.
  \item
    Encourages participation among all participants.
  \item
    Accepts criticism openly and non-defensively.
  \item
    Cooperates with others.
  \end{itemize}
\item
  Self-Management

  \begin{itemize}
  \tightlist
  \item
    Monitors progress to ensure that goals are met.
  \item
    Puts top priority on getting results.
  \item
    Defines task priorities for work sessions.
  \item
    Encourages others to express their views even when they are contrary.
  \item
    Stays focused on the task during meetings.
  \item
    Uses meeting time efficiently.
  \item
    Suggests ways to proceed during work sessions.
  \end{itemize}
\end{itemize}
